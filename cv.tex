%----------------------------------------------------------------------------------------
%	DOCUMENT DEFINITION
%----------------------------------------------------------------------------------------

% we use article class because we want to fully customize the page
\documentclass[10pt,A4]{article}	


%----------------------------------------------------------------------------------------
%	ENCODING
%----------------------------------------------------------------------------------------

%we use utf8 since we want to build from any machine
\usepackage[utf8]{inputenc}	

%----------------------------------------------------------------------------------------
%	LOGIC
%----------------------------------------------------------------------------------------
% For color
\usepackage{xcolor}

\usepackage{xifthen}
\usepackage{calc}
\usepackage[hidelinks]{hyperref}

%----------------------------------------------------------------------------------------
%	FONT
%----------------------------------------------------------------------------------------

% some tex-live fonts - choose your own

%\usepackage[defaultsans]{droidsans}
%\usepackage[default]{comfortaa}
%\usepackage{cmbright}
\usepackage[default]{raleway}
%\usepackage{fetamont}
%\usepackage[default]{gillius}
%\usepackage[light,math]{iwona}
%\usepackage[thin]{roboto} 

% set font default
\renewcommand*\familydefault{\sfdefault} 	
\usepackage[T1]{fontenc}

% more font size definitions
\usepackage{moresize}		

% font icons package
\usepackage{fontawesome}

%----------------------------------------------------------------------------------------
%	PAGE LAYOUT  DEFINITIONS
%----------------------------------------------------------------------------------------

%define page styles using geometry
\usepackage[a4paper]{geometry}		

% for example, change the margins to 2 inches all round
\geometry{top=2cm, bottom=2cm, left=2cm, right=2cm} 	

% use customized header
\usepackage{fancyhdr}				
\pagestyle{fancy}

%less space between header and content
\setlength{\headheight}{-5pt}		

% customize header entries
\lhead{}
\rhead{}
\chead{}

%indentation is zero
\setlength{\parindent}{0mm}

%----------------------------------------------------------------------------------------
%	GRAPHICS DEFINITIONS
%---------------------------------------------------------------------------------------- 

% for drawing graphics and charts
\usepackage{tikz}
\usetikzlibrary{shapes, backgrounds}

% use to vertically center content
% credits to: http://tex.stackexchange.com/questions/7219/how-to-vertically-center-two-images-next-to-each-other
\newcommand{\vcenteredinclude}[1]{\begingroup
	\setbox0=\hbox{\includegraphics{#1}}%
	\parbox{\wd0}{\box0}\endgroup}

% use to vertically center content
% credits to: http://tex.stackexchange.com/questions/7219/how-to-vertically-center-two-images-next-to-each-other
\newcommand*{\vcenteredhbox}[1]{\begingroup
	\setbox0=\hbox{#1}\parbox{\wd0}{\box0}\endgroup}

%----------------------------------------------------------------------------------------
%	ICON-SET EMBEDDING
%---------------------------------------------------------------------------------------- 

% at this point we simplify our icon-embedding by simply referring to a set of png images.
% if you find a good way of including svg without conflicting with other packages you can
% replace this part
\newcommand{\icon}[2]{\colorbox{black}{\makebox(#2, #2){\textcolor{white}{\large\csname fa#1\endcsname}}}}	%icon shortcut
\newcommand{\icontext}[3]{ 						%icon with text shortcut
	\vcenteredhbox{\icon{#1}{#2}}\hspace{0.2cm}\vcenteredhbox{\textcolor{black}{#3}}
}

%----------------------------------------------------------------------------------------
% 	HEADER
%----------------------------------------------------------------------------------------

% remove top header line
\renewcommand{\headrulewidth}{0pt} 

%remove botttom header line
\renewcommand{\footrulewidth}{0pt}	  	

%remove pagenum
\renewcommand{\thepage}{}	

%remove section num		
\renewcommand{\thesection}{}			


%----------------------------------------------------------------------------------------
%
% 	TIKZ GRAPHICS
%
%----------------------------------------------------------------------------------------

\newcounter{a}
\newcounter{b}
\newcounter{c}
\newcounter{barcount}

%----------------------------------------------------------------------------------------
% 	BAR CHART
%----------------------------------------------------------------------------------------

% draw a bar chart
% param 1: width
% param 2: height
% param 3: border color
% param 4: label text color
% param 5: label bg color
% param 6: cat 1 color
\newenvironment{barchart}[8]{
	
	\newcommand{\barwidth}{0.35}
	\newcommand{\barsep}{0.2}
	
	% param 1: overall percent
	% param 2: label
	% param 3: cat 1 percent
	% param 4: cat 2 percent
	% param 5: cat 3 percent
	\newcommand{\baritem}[5]{
		
		\pgfmathparse{##3+##4+##5}
		\let\perc\pgfmathresult
		
		\pgfmathparse{#2}
		\let\barsize\pgfmathresult
		
		\pgfmathparse{\barsize*##3/100}
		\let\barone\pgfmathresult
		
		\pgfmathparse{\barsize*##4/100}
		\let\bartwo\pgfmathresult
		
		\pgfmathparse{\barsize*##5/100}
		\let\barthree\pgfmathresult
		
		\pgfmathparse{(\barwidth*\thebarcount)+(\barsep*\thebarcount)}
		\let\barx\pgfmathresult
		
		\filldraw[fill=#6, draw=none] (0,-\barx) rectangle (\barone,-\barx-\barwidth);
		\filldraw[fill=#7, draw=none] (\barone, -\barx) rectangle (\barone+\bartwo,-\barx-\barwidth);
		\filldraw[fill=#8, draw=none] (\barone+\bartwo,-\barx ) rectangle (\barone+\bartwo+\barthree,-\barx-\barwidth);
		
		\node [label=180:\colorbox{#5}{\textcolor{#4}{##2}}] at (0,-\barx-0.175) {};
		\addtocounter{barcount}{1}
	}
	\begin{tikzpicture}
	\setcounter{barcount}{0}
	
}
{\end{tikzpicture}}

%----------------------------------------------------------------------------------------
% 	BUBBLE CHART
%----------------------------------------------------------------------------------------
\newcommand{\bubble}[5]{
	\definecolor{tmpcol}{RGB}{50,50,#5}
	% slice
	\filldraw[fill=black,draw=none] (#1,0.5) circle (#3);
	
	% outer label
	\node[label=\textcolor{black}{#4}] at (#1,0.7) {};
}

\newcommand{\bubbles}[2]{
	%reset counters
	\setcounter{a}{0}
	\setcounter{c}{150}
	\begin{tikzpicture}[scale=3]
	\foreach \p/\t in {#1} {
		\addtocounter{a}{1}
		\bubble{\thea/2}{\theb}{\p/25}{\t}{1\p0}
	}
	\end{tikzpicture}
}


%----------------------------------------------------------------------------------------
%	custom sections
%----------------------------------------------------------------------------------------

% create a coloured box with arrow and title as cv section headline
% param 1: section title
%
\newcommand{\cvsection}[1] {
	\textcolor{white}{\MakeUppercase{\textbf{#1}}}
}

\newcommand{\cvsect}[1]{
	\colorbox{black}{{\cvsection{#1}}}\\\\%
}

%----------------------------------------------------------------------------------------
% CUSTOM LOREM IPSUM
%----------------------------------------------------------------------------------------
\newcommand{\lorem}{Lorem ipsum dolor sit amet, consectetur adipiscing elit. Donec a diam lectus.}

%----------------------------------------------------------------------------------------
% ENTRY LIST
%----------------------------------------------------------------------------------------
\usepackage{tabularx}

\setlength{\tabcolsep}{0pt}
\newenvironment{entrylist}{%
	\begin{tabular*}{\textwidth}[t]{@{\extracolsep{\fill}}ll}
	}{%
	\end{tabular*}
}

\newcommand{\entry}[4]{%
	\parbox[t]{3.5cm}{%
		#1%
	}%
	&\parbox[t]{14cm}{%
		\textbf{#2}%
		\hfill%
		{\footnotesize \textbf{\textcolor{black}{#3}}}\\%
		#4%
	}\\\\}

\newcommand{\slashsep}{
	\hspace{2mm}/\hspace{2mm}
}

%----------------------------------------------------------------------------------------
%	DOCUMENT CONTENT
%----------------------------------------------------------------------------------------
\begin{document}
	
\definecolor{azure}{rgb}{0.0, 0.5, 1.0}	%----------------------------------------------------------------------------------------
	%	TITLE HEADLINE
	%----------------------------------------------------------------------------------------
	\begin{minipage}[t]{0.4\textwidth}\hrule height 0pt width 0pt%
		\colorbox{black}{{\HUGE\textcolor{white}{\textbf{\MakeUppercase{Ricardo}}}}}%
		\\
		\colorbox{black}{{\HUGE\textcolor{white}{\textbf{\MakeUppercase{Ruiz}}}}}%
		
	\end{minipage}%
	\begin{minipage}[t]{0.34\textwidth}\hrule height 0pt width 0pt%
		\small%
		\icontext{MapMarker}{12}{Mexico City, Mexico}\\
		\icontext{Globe}{12}{\href{https://firgun.me/author}{firgun.me/author}}\\
		\icontext{At}{12}{\href{mailto:gaanzz11@gmail.com}{gaanzz11@gmail.com}}\\	
	\end{minipage}%
	\begin{minipage}[t]{0.3\textwidth}\hrule height 0pt width 0pt%
		\small%
		\icontext{Gitlab}{12}{\href{https://gitlab.com/Ricardoengitlab}{gitlab.com/Ricardoengitlab}}\\
		\icontext{Twitter}{12}{\href{https://twitter.com/@Ricardoentuiter}{@Ricardoentuiter}}\\
		\icontext{Calendar}{12}{Exp grad date: \textbf{Jun, 2021}}\\
	\end{minipage}%
	
	% manage space by reducing font size
	\small%
	\vspace{0.5cm}
	
	%----------------------------------------------------------------------------------------
	%	SKILLS AND TECHNOLOGIES
	%----------------------------------------------------------------------------------------
	
	\cvsect{Projects}
	\begin{entrylist}
		\entry
		{Currently \\ Mexico City}
		{Full stack developer}
		{Ultra maratón sierra mixe}
		{This is a client's project on the automation of the processes involved in the realization of a marathon race such as registration, marketing and payment. \\
			\texttt{React}\slashsep
			\texttt{Nextjs}}
		\entry
		{Present \\ Mexico City}
		{Full stack developer}
		{Firgun}
		{This is a personal project where people can come to learn about programming languages, especially on the web. It also allows people to learn about curious things that programming languages have inside them in order to motivate people to learn. \\
			\texttt{React}\slashsep
			\texttt{Gatsby}\slashsep
			\texttt{GraphQL}}
		\entry
		{Present\\ Mexico City}
		{Full stack developer}
		{The Art of Design}
		{This is a personal project where I investigate the design of things or places in order to show the details that hide in their design and in the experience of use and thus be able to better understand the process of creating an object and how it interacts with people\\
			\texttt{React}\slashsep
			\texttt{Gatsby}\slashsep
			\texttt{GraphQL}}
		\entry
		{Autumn 2019\\ Mexico City}
		{Back-end developer}
		{Othello Game IA}
		{Game application to simulate the behavior of an artificial intelligence capable of play the Othello game and choose the best move depending on the level of difficulty chosen.\\
			\texttt{Processing}\slashsep
			\texttt{Java}}
		\entry
		{Summer 2018\\ Mexico City}
		{Full stack developer}
		{Copnap}
		{Mobile application for Android OS to help teachers and students to know if a number is prime or not, also the app can give you the list of prime numbers in a range and calculate the Euler's totient function.\\
			\texttt{Flutter}\slashsep
			\texttt{Dart}}






	\end{entrylist}
	\\
	\cvsect{Education}
	\begin{entrylist}
		\entry
		{2017– 2021 \\ Mexico City}
		{BS in Computer Science}
		{UNAM, Faculty of Science}
		{Relevant Coursework: Automata Theory, Computer Architecture, Algorithm Analysis, Artificial Intelligence, Software Engineering, Operating Systems, Data Structures, Relational Databases, Distributed Computing, Geographic Information Systems \\ 
		\texttt{\textbf{GPA: 81}}}
	\end{entrylist}
	\\
		
	

	\begin{minipage}[t]{0.5\textwidth}\hrule height 0pt width 0pt%
	\cvsect{Technical Skills}
	\begin{minipage}[t]{0.3\textwidth}%
		3 years\\ 2 years \\ 1 year \\ 6 months \\ 2 months \\ Currently
	\end{minipage}%
	\begin{minipage}[t]{0.7\textwidth}%
		\textbf{Python, Java}\\ \textbf{HTML, CSS, Haskell} \\ \textbf{Javascript, Flutter, Dart, Oracle SQL} \\ \textbf{C\#, Unity, Racket} \\ \textbf{C, Django, Assembly, React} \\ \textbf{Nextjs, Gatsby, GraphQL, Go}
	\end{minipage}%
	\end{minipage}%
	\hspace{1cm}
	\begin{minipage}[t]{0.45\textwidth}\hrule height 0pt width 0pt%
		\cvsect{Workflow}
		\textbf{Mac OS, Linux, VS Code, Gitlab, Netlify}
	\end{minipage}%
	\hspace{2cm}
	\\
	



	
	\begin{minipage}[t]{0.3\textwidth}\hrule height 0pt width 0pt%
		\cvsect{Languages}
		\textbf{Spanish} - native\\
		\textbf{English} - proficient\\
		\textbf{French} - limited\\
		\textbf{Portuguese} - limited\\
	\end{minipage}%
	\hspace{0cm}
	\begin{minipage}[t]{0.3\textwidth}\hrule height 0pt width 0pt%
		\cvsect{Hobbies}
		I love to try new technologies that help me to improve my productivity and create awesome things with them. 
		I like playing video and board games with my friends, especially Rocket League and Chess.
	\end{minipage}%
	\hspace{2cm}
	\begin{minipage}[t]{0.3\textwidth}\hrule height 0pt width 0pt%
		\cvsect{Interests}
		\texttt{Augmented reality}\slashsep
		\texttt{Machine Learning}\slashsep
		\texttt{Data Science}\slashsep
		\texttt{Computer Graphics}\slashsep
		\texttt{Open Source}\slashsep
		\texttt{UI}\slashsep
		\texttt{Cryptography}\slashsep
		\texttt{Concurrent systems}\slashsep
		\texttt{UX}\slashsep
		\texttt{Data Mining} \slashsep
		\texttt{Computer hardware}

		
	\end{minipage}%
	% LANGUAGES
	
\end{document}