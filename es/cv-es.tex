%----------------------------------------------------------------------------------------
%	DOCUMENT DEFINITION
%----------------------------------------------------------------------------------------

% we use article class because we want to fully customize the page
\documentclass[11pt,A4]{article}


%----------------------------------------------------------------------------------------
%	ENCODING
%----------------------------------------------------------------------------------------

%we use utf8 since we want to build from any machine
\usepackage[utf8]{inputenc}

%----------------------------------------------------------------------------------------
%	LOGIC
%----------------------------------------------------------------------------------------
% For color
\usepackage{xcolor}

\usepackage{xifthen}
\usepackage{calc}
\usepackage[hidelinks]{hyperref}

%----------------------------------------------------------------------------------------
%	FONT
%----------------------------------------------------------------------------------------

% some tex-live fonts - choose your own

%\usepackage[defaultsans]{droidsans}
%\usepackage[default]{comfortaa}
%\usepackage{cmbright}
\usepackage[default]{raleway}
%\usepackage{fetamont}
%\usepackage[default]{gillius}
%\usepackage[light,math]{iwona}
%\usepackage[thin]{roboto}

% set font default
\renewcommand*\familydefault{\sfdefault}
\usepackage[T1]{fontenc}

% more font size definitions
\usepackage{moresize}

% font icons package
\usepackage{fontawesome}

%----------------------------------------------------------------------------------------
%	PAGE LAYOUT  DEFINITIONS
%----------------------------------------------------------------------------------------

%define page styles using geometry
\usepackage[a4paper]{geometry}

% for example, change the margins to 2 inches all round
\geometry{top=2cm, bottom=2cm, left=2cm, right=2cm}

% use customized header
\usepackage{fancyhdr}
\pagestyle{fancy}

%less space between header and content
\setlength{\headheight}{-5pt}

% customize header entries
\lhead{}
\rhead{}
\chead{}

%indentation is zero
\setlength{\parindent}{0mm}

%----------------------------------------------------------------------------------------
%	GRAPHICS DEFINITIONS
%---------------------------------------------------------------------------------------- 

% for drawing graphics and charts
\usepackage{tikz}
\usetikzlibrary{shapes, backgrounds}

% use to vertically center content
% credits to: http://tex.stackexchange.com/questions/7219/how-to-vertically-center-two-images-next-to-each-other
\newcommand{\vcenteredinclude}[1]{\begingroup
	\setbox0=\hbox{\includegraphics{#1}}%
	\parbox{\wd0}{\box0}\endgroup}

% use to vertically center content
% credits to: http://tex.stackexchange.com/questions/7219/how-to-vertically-center-two-images-next-to-each-other
\newcommand*{\vcenteredhbox}[1]{\begingroup
	\setbox0=\hbox{#1}\parbox{\wd0}{\box0}\endgroup}

%----------------------------------------------------------------------------------------
%	ICON-SET EMBEDDING
%---------------------------------------------------------------------------------------- 

% at this point we simplify our icon-embedding by simply referring to a set of png images.
% if you find a good way of including svg without conflicting with other packages you can
% replace this part
\newcommand{\icon}[2]{\colorbox{black}{\makebox(#2, #2){\textcolor{white}{\large\csname fa#1\endcsname}}}}	%icon shortcut
\newcommand{\icontext}[3]{ 						%icon with text shortcut
	\vcenteredhbox{\icon{#1}{#2}}\hspace{0.2cm}\vcenteredhbox{\textcolor{black}{#3}}
}

%----------------------------------------------------------------------------------------
% 	HEADER
%----------------------------------------------------------------------------------------

% remove top header line
\renewcommand{\headrulewidth}{0pt}

%remove botttom header line
\renewcommand{\footrulewidth}{0pt}

%remove pagenum
\renewcommand{\thepage}{}

%remove section num
\renewcommand{\thesection}{}


%----------------------------------------------------------------------------------------
%
% 	TIKZ GRAPHICS
%
%----------------------------------------------------------------------------------------

\newcounter{a}
\newcounter{b}
\newcounter{c}
\newcounter{barcount}

%----------------------------------------------------------------------------------------
% 	BAR CHART
%----------------------------------------------------------------------------------------

% draw a bar chart
% param 1: width
% param 2: height
% param 3: border color
% param 4: label text color
% param 5: label bg color
% param 6: cat 1 color
\newenvironment{barchart}[8]{

	\newcommand{\barwidth}{0.35}
	\newcommand{\barsep}{0.2}

	% param 1: overall percent
	% param 2: label
	% param 3: cat 1 percent
	% param 4: cat 2 percent
	% param 5: cat 3 percent
	\newcommand{\baritem}[5]{

		\pgfmathparse{##3+##4+##5}
		\let\perc\pgfmathresult

		\pgfmathparse{#2}
		\let\barsize\pgfmathresult

		\pgfmathparse{\barsize*##3/100}
		\let\barone\pgfmathresult

		\pgfmathparse{\barsize*##4/100}
		\let\bartwo\pgfmathresult

		\pgfmathparse{\barsize*##5/100}
		\let\barthree\pgfmathresult

		\pgfmathparse{(\barwidth*\thebarcount)+(\barsep*\thebarcount)}
		\let\barx\pgfmathresult

		\filldraw[fill=#6, draw=none] (0,-\barx) rectangle (\barone,-\barx-\barwidth);
		\filldraw[fill=#7, draw=none] (\barone, -\barx) rectangle (\barone+\bartwo,-\barx-\barwidth);
		\filldraw[fill=#8, draw=none] (\barone+\bartwo,-\barx ) rectangle (\barone+\bartwo+\barthree,-\barx-\barwidth);

		\node [label=180:\colorbox{#5}{\textcolor{#4}{##2}}] at (0,-\barx-0.175) {};
		\addtocounter{barcount}{1}
	}
	\begin{tikzpicture}
	\setcounter{barcount}{0}
}
{\end{tikzpicture}}

%----------------------------------------------------------------------------------------
% 	BUBBLE CHART
%----------------------------------------------------------------------------------------
\newcommand{\bubble}[5]{
	\definecolor{tmpcol}{RGB}{50,50,#5}
	% slice
	\filldraw[fill=black,draw=none] (#1,0.5) circle (#3);
	% outer label
	\node[label=\textcolor{black}{#4}] at (#1,0.7) {};
}

\newcommand{\bubbles}[2]{
	%reset counters
	\setcounter{a}{0}
	\setcounter{c}{150}
	\begin{tikzpicture}[scale=3]
	\foreach \p/\t in {#1} {
		\addtocounter{a}{1}
		\bubble{\thea/2}{\theb}{\p/25}{\t}{1\p0}
	}
	\end{tikzpicture}
}


%----------------------------------------------------------------------------------------
%	custom sections
%----------------------------------------------------------------------------------------

% create a coloured box with arrow and title as cv section headline
% param 1: section title
%
\newcommand{\cvsection}[1] {
	\textcolor{white}{\MakeUppercase{\textbf{#1}}}
}

\newcommand{\cvsect}[1]{
	\colorbox{black}{{\cvsection{#1}}}\\\\%
}

%----------------------------------------------------------------------------------------
% CUSTOM LOREM IPSUM
%----------------------------------------------------------------------------------------
\newcommand{\lorem}{Lorem ipsum dolor sit amet, consectetur adipiscing elit. Donec a diam lectus.}

%----------------------------------------------------------------------------------------
% ENTRY LIST
%----------------------------------------------------------------------------------------
\usepackage{tabularx}

\setlength{\tabcolsep}{0pt}
\newenvironment{entrylist}{%
	\begin{tabular*}{\textwidth}[t]{@{\extracolsep{\fill}}ll}
	}{%
	\end{tabular*}
}

\newcommand{\entry}[5]{%
	&\parbox[t]{17.5cm}{%
		\large\textbf{#1}%
		\hfill
		{\small \textbf{\textcolor{black}{#2}}}\\%
		\normalsize #4\\
		\texttt{#5}
%		\hfill%%
%		{\footnotesize \textbf{\textcolor{black}{#3}}}\\%
%		#4%
%		{\textbf{#4}}
	}\\\\}

\newcommand{\slashsep}{
	\hspace{2mm}/\hspace{2mm}
}

%----------------------------------------------------------------------------------------
%	DOCUMENT CONTENT
%----------------------------------------------------------------------------------------
\begin{document}

\definecolor{azure}{rgb}{0.0, 0.5, 1.0}	%----------------------------------------------------------------------------------------
	%	TITLE HEADLINE
	%----------------------------------------------------------------------------------------
	\begin{minipage}[t]{0.37\textwidth}\hrule height 0pt width 0pt%
		\colorbox{black}{{\HUGE\textcolor{white}{\textbf{\MakeUppercase{Ricardo}}}}}%
		\\
		\colorbox{black}{{\HUGE\textcolor{white}{\textbf{\MakeUppercase{Ruiz}}}}}%

	\end{minipage}%
	\begin{minipage}[t]{0.35\textwidth}\hrule height 0pt width 0pt%
		\small%
		\icontext{MapMarker}{12}{Mexico City, Mexico}\\

		\icontext{Linkedin}{12}{\href{https://linkedin.com/in/ricardo-ruiz-194b531aa}{\underline{Ricardo Ruiz}}}\\

		\icontext{At}{12}{\href{mailto:gaanzz11@gmail.com}{gaanzz11@gmail.com}}\\
	\end{minipage}%
	\begin{minipage}[t]{0.4\textwidth}\hrule height 0pt width 0pt%
		\small%
		\icontext{Gitlab}{12}{\href{https://gitlab.com/Ricardoengitlab}{gitlab.com/Ricardoengitlab}}\\

		\icontext{Github}{12}{\href{https://github.com/Ricardoengithub}{github.com/Ricardoengithub}}\\

		\icontext{Phone}{12}{\href{}{(+52) 5620278551}}\\

		%\icontext{Calendar}{12}{Exp grad date: \textbf{Jun, 2022}}\\
	\end{minipage}%

	% manage space by reducing font size
	\vspace{0.1cm}
	%----------------------------------------------------------------------------------------
	%	SKILLS AND TECHNOLOGIES
	%----------------------------------------------------------------------------------------

	\cvsect{Resumen}
	\normalsize Tengo 6 años de experiencia en programación y 2 años de experiencia profesional en sectores como seguros, bienes raíces e e-commerce. Trabajo principalmente con Javascript(con experiencia en React), Python(con experiencia en Django) y Java. Gracias a mi formación puedo desenvolverme en cualquier área de desarrollo incluyendo la investigación sin embargo estoy interesado principalmente en la parte del front. \\
	\\
	\cvsect{Experiencia}
	\begin{entrylist}
		\entry
		{Front end developer en GNP Seguros}
		{Abril 2021 - Febrero 2022}
		{GNP Seguros}
		{Trabajé creando la UI  donde los agentes llenan los datos 		          necesarios para obtener la cotización del seguro. Junto a los equipos de BE y DB coordiné las integraciones con el equipo de Front así como la asignación de actividades a desempeñar por parte del equipo.}
		{ React \slashsep Redux \slashsep Material UI \slashsep Team Management}
		\entry
		{Full stack developer en M22}
		{Octubre 2020 - Febrero 2021}
		{M22}
		{Realicé la UI donde las personas pueden ver en tiempo real los detalles y la disponibilidad de las posibles opciones de compra en distintos tipos de proyectos. Además de la integración con un CMS(Strapi) para la actualización y administración de la información por parte de la empresa.}
		{React \slashsep Strapi \slashsep QGIS \slashsep GraphQL \slashsep PostgreSQL}
		\entry
		{Full stack developer en Ultra maratón sierra mixe}
		{Enero 2020 - Septiembre 2020}
		{Ultra maratón sierra mixe}
		{Realicé la página web de la carrera así como la automatización de los procesos que involucran su realización, tales como la publicidad, el registro y el pago.}
		{React \slashsep Google Firebase \slashsep Bootstrap}
		\entry
		{Full stack developer}
		{Verano 2018}
		{Copnap}
		{Realicé una aplicación para Android donde ayudo a los profesores a calcular la función $\phi$ de Euler, además de poder calcular si un número es primo y los números primos dentro de un rango dado.}
		{Android Studio \slashsep Flutter \slashsep Dart}
	\end{entrylist}
	\\
	\cvsect{Educación}
	\begin{entrylist}
		\entry
		{Ciencias de la computación (\small 2017 - 2022)}
		{Universidad Nacional Autónoma de México }
		{UNAM, Facultad de ciencias}
		{\underline{Cursos relevantes:} Autómatas, Arquitectura de computadoras, Análisis de algoritmos, Inteligencia Artificial, Ingeniería de software, Sistemas operativos, Estructura de datos, Bases de datos relacionales, Computación distribuida, Sistemas de información geográfica} 
		{GPA: 85}
	\end{entrylist}
	\\
	\begin{minipage}[t]{0.65\textwidth}\hrule height 0pt width 0pt%
	\cvsect{Habilidades}
	\begin{minipage}[t]{0.2\textwidth}%
		4 años\\ 3 años\\ 2 años\\ 1 años\\ 6 meses
	\end{minipage}%
	\begin{minipage}[t]{0.7\textwidth}%
		\textbf{Python, Java, HTML, CSS} \\ \textbf{React, Javascript, Flutter, Dart, Oracle SQL} \\ \textbf{C\#, Unity, Racket, C, Django, Haskell} \\ \textbf{Gatsby, GraphQL, Redux} \\ \textbf{Nextjs, Golang, Strapi}
	\end{minipage}%
	\end{minipage}%
	\hspace{1cm}
	\begin{minipage}[t]{0.35\textwidth}\hrule height 0pt width 0pt%
		\cvsect{Idiomas}
		\textbf{Español} - Nativo\\
		\textbf{Inglés} - Conversacional\\
		\textbf{Francés} - Intermedio\\
	\end{minipage}%

\end{document}